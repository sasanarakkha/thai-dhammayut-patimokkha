\section{Saṅkhittapātimokkh'uddeso}
\label{sankhittapatimokkh'uddeso}
% TODO: append pali word for conclusion
Uddiṭṭhaṁ kho āyasmanto nidānaṁ, uddiṭṭhā cattāro pārājikā dhammā, uddiṭṭhā terasa saṅghādisesā dhammā, uddiṭṭhā dve aniyatā dhammā. Sutā kho pan'āyasmantehi terasa saṅghādisesā dhammā, dve aniyatā dhammā, tiṁsa nissaggiyā pācittiyā dhammā, dvenavuti pācittiyā dhammā, cattāro pāṭidesanīyā dhammā, pañcasattati sekhiyā dhammā, satta adhikaraṇasamathā dhammā, ettakaṁ tassa bhagavato sutt'āgataṁ suttapariyāpannaṁ, anvaḍḍhamāsaṁ uddesaṁ āgacchati, tattha sabbeh'eva samaggehi sammodamānehi avivadamānehi sikkhitabban-ti.

\begin{outro}
Bhikkhupātimokkhaṁ niṭṭhitaṁ
\end{outro}

\clearpage

\section{The Pātimokkha Recitation in Brief Conclusion}
\label{patimokkha-in-brief-conclusion}

Venerables, the introduction has been recited, the four cases involving disqualification have been recited, the thirteen cases [involving] the community in the beginning and in the rest have been recited, the two indefinite cases have been recited. Heard by the venerables have been the thirty cases involving expiation with forfeiture, the ninety-two cases involving expiation, the four cases that are to be acknowledged, the cases related to the training, the seven cases that are settlements of legal issues. This much [training-rule] of the Fortunate One has been handed down in the Sutta, has been included in the Sutta, comes up for recitation half-monthly. By all who are united, who are on friendly terms, who are not disputing, is to trained herein.

\begin{outro}
  The Bhikkhu Disciplinary Code has been finished
\end{outro}

\clearpage
