\section{Acknowledgments}
\label{ack}

\begin{center}
  Venerables, these four cases that are to be acknowledged come up for recitation.
\end{center}

\setsubsecheadstyle{\subsubsectionFmt}
\pdfbookmark[3]{Acknowledgment 1}{ack1}
\subsection*{\hyperref[pd1]{Acknowledgment 1: The first training precept on what is to be acknowledged}}
\label{ack1}
If any bhikkhu, having accepted [it] with his own hand from the hand of an unrelated bhikkhunī who has entered an inhabited area [for alms], should chew uncooked food or eat cooked food], [it] is to be acknowledged by that bhikkhu [saying]: ``Friend[s], I have committed a blameworthy act which is unsuitable, which is to be acknowledged; I acknowledge it.''

\pdfbookmark[3]{Acknowledgment 2}{ack2}
\subsection*{\hyperref[pd2]{Acknowledgment 2: The second training precept on what is to be acknowledged}}
\label{ack2}
Now, bhikkhus who have been invited are eating among families, and if a bhikkhunī who is giving directions is standing there [saying], ``Give curry here, give rice here!'' [then] by those bhikkhus that bhikkhunī is to be dismissed [saying], ``Go away, sister, for as long as the bhikkhus eat!,'' and if not even one bhikkhu would speak against [it, so as] to dismiss that bhikkhunī [saying], ``Go away, sister, for as long as the bhikkhus eat!,'' [then it] is to be acknowledged by those bhikkhus, ``Friend[s], we have committed a blameworthy act which is unsuitable, which is to be acknowledged; we acknowledge it.''

\pdfbookmark[3]{Acknowledgment 3}{ack3}
\subsection*{\hyperref[pd3]{Acknowledgment 3: The third training precept on what is to be acknowledged}}
\label{ack3}
Now, [there are] those families which are agreed upon as trainees: if any bhikkhu who has not been invited beforehand, who is not ill, should chew uncooked food or eat cooked food having accepted [it] with his own hand in families who are of such a kind, who are considered trainees, [then it] is to be acknowledged by that bhikkhu: ``Friend[s], I have committed a blameworthy act which is unsuitable, which is to be acknowledged; I acknowledge it.''

\pdfbookmark[3]{Acknowledgment 4}{ack4}
\subsection*{\hyperref[pd4]{Acknowledgment 4: The fourth training precept on what is to be acknowledged}}
\label{ack4}
Now, [there are] those those wilderness lodgings which are considered risky, which are dangerous: if any bhikkhu, [staying] in lodgings which are of such a kind, without having announced [the danger] beforehand, having accepted [the food] with his own hand inside the monastery, [and then] not being ill, should chew uncooked food or eat cooked food, [then it] is to be acknowledged by that bhikkhu, ``Friend[s], I have committed a blameworthy act which is unsuitable, which is to be acknowledged; I acknowledge it.''

\medskip

\begin{center}
Venerables, the four cases that are to be acknowledged have been recited.

\smallskip

Concerning that I ask the Venerables: [Are you] pure in this?\\
A second time again I ask: [Are you] pure in this?\\
A third time again I ask: [Are you] pure in this?

\smallskip

The venerables are pure in this, therefore there is silence, thus I bear this [in mind].
\end{center}

\begin{outro}
  The [cases] which are to be acknowledged have finished
\end{outro}

\clearpage
