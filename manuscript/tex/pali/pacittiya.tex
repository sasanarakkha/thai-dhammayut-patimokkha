\section{Pācittiyā}
\label{pc}

\begin{intro}
  Ime kho pan'āyasmanto dve-navuti pācittiyā dhammā uddesaṁ āgacchanti.
\end{intro}

\subsection{Musāvādavaggo}
\vspace{0.2cm}

\pdfbookmark[3]{Pācittiya 1}{pac1}
\subsubsection*{\hyperref[exp1]{Pācittiya 1: Musāvādasikkhāpadaṁ}}
\label{pac1}

Sampajāna-musāvāde pācittiyaṁ.

\pdfbookmark[3]{Pācittiya 2}{pac2}
\subsubsection*{\hyperref[exp2]{Pācittiya 2: Omasavādasikkhāpadaṁ}}
\label{pac2}

Omasavāde pācittiyaṁ.

\pdfbookmark[3]{Pācittiya 3}{pac3}
\subsubsection*{\hyperref[exp3]{Pācittiya 3: Pesuññasikkhāpadaṁ}}
\label{pac3}

Bhikkhu-pesuññe pācittiyaṁ.

\pdfbookmark[3]{Pācittiya 4}{exp4}
\subsubsection*{\hyperref[exp4]{Pācittiya 4: Padasodhammasikkhāpadaṁ}}
\label{pac4}

Yo pana bhikkhu anupasampannaṁ padaso dhammaṁ vāceyya, pācittiyaṁ.

\pdfbookmark[3]{Pācittiya 5}{pac5}
\subsubsection*{\hyperref[exp5]{Pācittiya 5: Paṭhamasahaseyyasikkhāpadaṁ}}
\label{pac5}

Yo pana bhikkhu anupasampannena uttari-dviratta-tirattaṁ saha-seyyaṁ kappeyya, pācittiyaṁ.

\pdfbookmark[3]{Pācittiya 6}{pac6}
\subsubsection*{\hyperref[exp6]{Pācittiya 6: Dutiyasahaseyyasikkhāpadaṁ}}
\label{pac6}

Yo pana bhikkhu mātugāmena saha-seyyaṁ kappeyya, pācittiyaṁ.

\pdfbookmark[3]{Pācittiya 7}{pac7}
\subsubsection*{\hyperref[exp7]{Pācittiya 7: Dhammadesanāsikkhāpadaṁ}}
\label{pac7}

Yo pana bhikkhu mātugāmassa uttari-chappañca-vācāhi dhammaṁ deseyya, aññatra viññunā purisa-viggahena, pācittiyaṁ.

\pdfbookmark[3]{Pācittiya 8}{pac8}
\subsubsection*{\hyperref[exp8]{Pācittiya 8: Bhūtārocanasikkhāpadaṁ}}
\label{pac8}

Yo pana bhikkhu anupasampannassa uttari-manussa-dhammaṁ āroceyya, bhūtasmiṁ pācittiyaṁ.

\pdfbookmark[3]{Pācittiya 9}{pac9}
\subsubsection*{\hyperref[exp9]{Pācittiya 9: Duṭṭhullārocanasikkhāpadaṁ}}
\label{pac9}

Yo pana bhikkhu bhikkhussa duṭṭhullaṁ āpattiṁ anupasampannassa āroceyya aññatra bhikkhu-sammatiyā, pācittiyaṁ.

\pdfbookmark[3]{Pācittiya 10}{pac10}
\subsubsection*{\hyperref[exp10]{Pācittiya 10: Paṭhavīkhaṇanasikkhāpadaṁ}}
\label{pac10}

Yo pana bhikkhu paṭhaviṁ khaṇeyya vā khaṇāpeyya vā, pācittiyaṁ.

\begin{center}
  Musāvāda-vaggo paṭhamo
\end{center}

\subsection{Bhūtagāmavaggo}
\vspace{0.2cm}

\pdfbookmark[3]{Pācittiya 11}{pac11}
\subsubsection*{\hyperref[exp11]{Pācittiya 11: Bhūtagāmasikkhāpadaṁ}}
\label{pac11}

Bhūtagāma-pātabyatāya pācittiyaṁ.

\pdfbookmark[3]{Pācittiya 12}{pac12}
\subsubsection*{\hyperref[exp12]{Pācittiya 12: Aññavādakasikkhāpadaṁ}}
\label{pac12}

Aññavādake vihesake pācittiyaṁ.

\pdfbookmark[3]{Pācittiya 13}{pac13}
\subsubsection*{\hyperref[exp13]{Pācittiya 13: Ujjhāpanakasikkhāpadaṁ}}
\label{pac13}

Ujjhāpanake khiyyanake pācittiyaṁ.

\pdfbookmark[3]{Pācittiya 14}{pac14}
\subsubsection*{\hyperref[exp14]{Pācittiya 14: Paṭhamasen'āsanasikkhāpadaṁ}}
\label{pac14}

Yo pana bhikkhu saṅghikaṁ mañcaṁ vā pīṭhaṁ vā bhisiṁ vā kocchaṁ vā ajjhokāse santharitvā vā santharāpetvā vā, taṁ pakkamanto n'eva uddhareyya na uddharāpeyya, anāpucchaṁ vā gaccheyya, pācittiyaṁ.

\pdfbookmark[3]{Pācittiya 15}{pac15}
\subsubsection*{\hyperref[exp15]{Pācittiya 15: Dutiyasen'āsanasikkhāpadaṁ}}
\label{pac15}

Yo pana bhikkhu saṅghike vihāre seyyaṁ santharitvā vā santharāpetvā vā, taṁ pakkamanto n'eva uddhareyya na uddharāpeyya, anāpucchaṁ vā gaccheyya, pācittiyaṁ.

\pdfbookmark[3]{Pācittiya 16}{pac16}
\subsubsection*{\hyperref[exp16]{Pācittiya 16: Anupakhajjasikkhāpadaṁ}}
\label{pac16}

Yo pana bhikkhu saṅghike vihāre jānaṁ pubbūpagataṁ bhikkhuṁ anūpakhajja seyyaṁ kappeyya, “Yassa sambādho bhavissati, so pakkamissatī” ti. Etad'eva paccayaṁ karitvā anaññaṁ, pācittiyaṁ.

\pdfbookmark[3]{Pācittiya 17}{pac17}
\subsubsection*{\hyperref[exp17]{Pācittiya 17: Nikkaḍḍhanasikkhāpadaṁ}}
\label{pac17}

Yo pana bhikkhu bhikkhuṁ kupito anattamano saṅghikā vihārā nikkaḍḍheyya vā nikkaḍḍhāpeyya vā, pācittiyaṁ.

\pdfbookmark[3]{Pācittiya 18}{pac18}
\subsubsection*{\hyperref[exp18]{Pācittiya 18: Vehāsakuṭisikkhāpadaṁ}}
\label{pac18}

Yo pana bhikkhu saṅghike vihāre upari-vehāsa-kuṭiyā āhacca-pādakaṁ mañcaṁ vā pīṭhaṁ vā abhinisīdeyya vā abhinipajjeyya vā, pācittiyaṁ.

\pdfbookmark[3]{Pācittiya 19}{pac19}
\subsubsection*{\hyperref[exp19]{Pācittiya 19: Mahallakavihārasikkhāpadaṁ}}
\label{pac19}

Mahallakam-pana bhikkhunā vihāraṁ kārayamānena, yāva dvāra-kosā aggalaṭṭhapanāya, āloka-sandhi-parikammāya, dvitticchadanassa pariyāyaṁ, appaharite ṭhitena adhiṭṭhātabbaṁ. Tato ce uttariṁ appaharite'pi ṭhito adhiṭṭhaheyya, pācittiyaṁ.

\pdfbookmark[3]{Pācittiya 20}{pac20}
\subsubsection*{\hyperref[exp20]{Pācittiya 20: Sappāṇakasikkhāpadaṁ}}
\label{pac20}

Yo pana bhikkhu jānaṁ sappāṇakaṁ udakaṁ tiṇaṁ vā mattikaṁ vā siñceyya vā siñcāpeyya vā, pācittiyaṁ.

\begin{center}
  Bhūtagāma-vaggo dutiyo
\end{center}

\subsection{Bhikkhunovādavaggo}
\vspace{0.2cm}

\pdfbookmark[3]{Pācittiya 21}{pac21}
\subsubsection*{\hyperref[exp]{Pācittiya 21: Ovādasikkhāpadaṁ}}
\label{pac21}

Yo pana bhikkhu asammato bhikkhuniyo ovadeyya, pācittiyaṁ.

\pdfbookmark[3]{Pācittiya 22}{pac22}
\subsubsection*{\hyperref[exp22]{Pācittiya 22: Atthaṅgatasikkhāpadaṁ}}
\label{pac22}

Sammato'pi ce bhikkhu atthaṅgate suriye bhikkhuniyo ovadeyya, pācittiyaṁ.

\pdfbookmark[3]{Pācittiya 23}{pac23}
\subsubsection*{\hyperref[exp23]{Pācittiya 23: Bhikkhunupassayasikkhāpadaṁ}}
\label{pac23}

Yo pana bhikkhu bhikkhunūpassayaṁ upasaṅkamitvā bhikkhuniyo ovadeyya aññatra samayā, pācittiyaṁ. Tatth'āyaṁ samayo: Gilānā hoti bhikkhunī. Ayaṁ tattha samayo.

\pdfbookmark[3]{Pācittiya 24}{pac24}
\subsubsection*{\hyperref[exp24]{Pācittiya 24: Āmisasikkhāpadaṁ}}
\label{pac24}
Yo pana bhikkhu evaṁ vadeyya, “Āmisa-hetu bhikkhū bhikkhuniyo ovadantī” ti, pācittiyaṁ.

\pdfbookmark[3]{Pācittiya 25}{pac25}
\subsubsection*{\hyperref[exp25]{Pācittiya 25: Cīvaradānasikkhāpadaṁ}}
\label{pac25}

Yo pana bhikkhu aññātikāya bhikkhuniyā cīvaraṁ dadeyya, aññatra pārivaṭṭakā, pācittiyaṁ.

\pdfbookmark[3]{Pācittiya 26}{pac26}
\subsubsection*{\hyperref[exp26]{Pācittiya 26: Cīvarasibbanasikkhāpadaṁ}}
\label{pac26}

Yo pana bhikkhu aññātikāya bhikkhuniyā cīvaraṁ sibbeyya vā sibbāpeyya vā, pācittiyaṁ.

\pdfbookmark[3]{Pācittiya 27}{pac27}
\subsubsection*{\hyperref[exp27]{Pācittiya 27: Saṁvidhānasikkhāpadaṁ}}
\label{pac27}

Yo pana bhikkhu bhikkhuniyā saddhiṁ saṁvidhāya ek'addhāna-maggaṁ paṭipajjeyya, antamaso gām'antaram'pi aññatra samayā, pācittiyaṁ. Tatth'āyaṁ samayo: Sattha-gamanīyo hoti maggo sāsaṅka-sammato sappaṭibhayo. Ayaṁ tattha samayo.

\pdfbookmark[3]{Pācittiya 28}{pac28}
\subsubsection*{\hyperref[exp28]{Pācittiya 28: Nāvābhiruhanasikkhāpadaṁ}}
\label{pac28}

Yo pana bhikkhu bhikkhuniyā saddhiṁ saṁvidhāya ekaṁ nāvaṁ abhirūheyya, uddha-gāminiṁ vā adho-gāminiṁ vā, aññatra tiriy'antaraṇāya, pācittiyaṁ.

\pdfbookmark[3]{Pācittiya 29}{pac29}
\subsubsection*{\hyperref[exp29]{Pācittiya 29: Paripācitasikkhāpadaṁ}}
\label{pac29}

Yo pana bhikkhu jānaṁ bhikkhunī-paripācitaṁ piṇḍapātaṁ bhuñjeyya, aññatra pubbe gihi-samārambhā, pācittiyaṁ.

\pdfbookmark[3]{Pācittiya 30}{pac30}
\subsubsection*{\hyperref[exp30]{Pācittiya 30: Rahonisajjasikkhāpadaṁ}}
\label{pac30}

Yo pana bhikkhu bhikkhuniyā saddhiṁ eko ekāya raho nisajjaṁ kappeyya, pācittiyaṁ.

\begin{center}
  Ovāda-vaggo tatiyo
\end{center}

\subsection{Bhojanavaggo}
\vspace{0.2cm}

\pdfbookmark[3]{Pācittiya 31}{pac31}
\subsubsection*{\hyperref[exp31]{Pācittiya 31: Āvasathapiṇḍasikkhāpadaṁ}}
\label{pac31}

Agilānena bhikkhunā eko āvasatha-piṇḍo bhuñjitabbo. Tato ce uttariṁ bhuñjeyya, pācittiyaṁ.

\pdfbookmark[3]{Pācittiya 32}{pac31}
\subsubsection*{\hyperref[exp32]{Pācittiya 32: Gaṇabhojanasikkhāpadaṁ}}
\label{pac32}

Gaṇa-bhojane aññatra samayā, pācittiyaṁ. Tatth'āyaṁ samayo: gilāna-samayo, cīvara-dāna-samayo, cīvara-kāra-samayo, addhāna-gamana-samayo, nāvābhirūhana-samayo, mahā-samayo, samaṇa-bhatta-samayo. Ayaṁ tattha samayo.

\pdfbookmark[3]{Pācittiya 33}{pac33}
\subsubsection*{\hyperref[exp33]{Pācittiya 33: Paramparabhojanasikkhāpadaṁ}}
\label{pac33}

Parampara-bhojane aññatra samayā, pācittiyaṁ. Tatth'āyaṁ samayo: gilāna-samayo, cīvara-dāna-samayo, cīvara-kāra-samayo. Ayaṁ tattha samayo.

\pdfbookmark[3]{Pācittiya 34}{pac34}
\subsubsection*{\hyperref[exp34]{Pācittiya 34: Kāṇamātusikkhāpadaṁ}}
\label{pac34}

Bhikkhuṁ pan'eva kulaṁ upagataṁ pūvehi vā manthehi vā abhihaṭṭhum-pavāreyya, ākaṅkhamānena bhikkhunā dvitti-patta-pūrā paṭiggahetabbā. Tato ce uttariṁ paṭiggaṇheyya, pācittiyaṁ. Dvitti-patta-pūre paṭiggahetvā tato nīharitvā bhikkhūhi saddhiṁ saṁvibhajitabbaṁ. Ayaṁ tattha sāmīci.

\pdfbookmark[3]{Pācittiya 35}{pac35}
\subsubsection*{\hyperref[exp35]{Pācittiya 35: Paṭhamapavāraṇāsikkhāpadaṁ}}
\label{pac35}

Yo pana bhikkhu bhuttāvī pavārito anatirittaṁ khādanīyaṁ vā bhojanīyaṁ vā khādeyya vā bhuñjeyya vā, pācittiyaṁ.

\pdfbookmark[3]{Pācittiya 36}{pac36}
\subsubsection*{\hyperref[exp36]{Pācittiya 36: Dutiyapavāraṇāsikkhāpadaṁ}}
\label{pac36}

Yo pana bhikkhu bhikkhuṁ bhuttāviṁ pavāritaṁ anatirittena khādanīyena vā bhojanīyena vā abhihaṭṭhum-pavāreyya, “Handa bhikkhu khāda vā bhuñja vā” ti, jānaṁ āsādan'āpekkho, bhuttasmiṁ pācittiyaṁ.

\pdfbookmark[3]{Pācittiya 37}{pac37}
\subsubsection*{\hyperref[exp37]{Pācittiya 37: Vikālabhojanasikkhāpadaṁ}}
\label{pac37}

Yo pana bhikkhu vikāle khādanīyaṁ vā bhojanīyaṁ vā khādeyya vā bhuñjeyya vā, pācittiyaṁ.

\pdfbookmark[3]{Pācittiya 38}{pac38}
\subsubsection*{\hyperref[exp38]{Pācittiya 38: Sannidhikārakasikkhāpadaṁ}}
\label{pac38}

Yo pana bhikkhu sannidhi-kārakaṁ khādanīyaṁ vā bhojanīyaṁ vā khādeyya vā bhuñjeyya vā, pācittiyaṁ.

\pdfbookmark[3]{Pācittiya 39}{pac39}
\subsubsection*{\hyperref[exp39]{Pācittiya 39: Paṇītabhojanasikkhāpadaṁ}}
\label{pac39}

Yāni kho pana tāni paṇīta-bhojanāni, seyyathīdaṁ: sappi navanītaṁ telaṁ madhu phāṇitaṁ, maccho maṁsaṁ khīraṁ dadhi. Yo pana bhikkhu eva-rūpāni paṇīta-bhojanāni agilāno attano atthāya viññāpetvā bhuñjeyya, pācittiyaṁ.

\pdfbookmark[3]{Pācittiya 40}{pac40}
\subsubsection*{\hyperref[exp40]{Pācittiya 40: Dantaponasikkhāpadaṁ}}
\label{pac40}

Yo pana bhikkhu adinnaṁ mukha-dvāraṁ āhāraṁ āhareyya, aññatra udaka-dantapoṇā, pācittiyaṁ.

\begin{center}
  Bhojana-vaggo catuttho
\end{center}

\subsection{Acelakavaggo}
\vspace{0.2cm}

\pdfbookmark[3]{Pācittiya 41}{pac41}
\subsubsection*{\hyperref[exp41]{Pācittiya 41: Acelakasikkhāpadaṁ}}
\label{pac41}

Yo pana bhikkhu acelakassa vā paribbājakassa vā paribbājikāya vā sahatthā khādanīyaṁ vā bhojanīyaṁ vā dadeyya, pācittiyaṁ.

\pdfbookmark[3]{Pācittiya 42}{pac42}
\subsubsection*{\hyperref[exp42]{Pācittiya 42: Uyyojanasikkhāpadaṁ}}
\label{pac42}

Yo pana bhikkhu bhikkhuṁ evaṁ vadeyya: “Eh'āvuso gāmaṁ vā nigamaṁ vā piṇḍāya pavisissāmā” ti. Tassa dāpetvā vā adāpetvā vā uyyojeyya, “Gacch'āvuso. Na me tayā saddhiṁ kathā vā nisajjā vā phāsu hoti. Ekakassa me kathā vā nisajjā vā phāsu hotī” ti. Etad'eva paccayaṁ karitvā anaññaṁ, pācittiyaṁ.

\pdfbookmark[3]{Pācittiya 43}{pac43}
\subsubsection*{\hyperref[exp43]{Pācittiya 43: Sabhojanasikkhāpadaṁ}}
\label{pac43}

Yo pana bhikkhu sabhojane kule anūpakhajja nisajjaṁ kappeyya, pācittiyaṁ.

\pdfbookmark[3]{Pācittiya 44}{pac44}
\subsubsection*{\hyperref[exp44]{Pācittiya 44: Rahopaṭicchannasikkhāpadaṁ}}
\label{pac44}

Yo pana bhikkhu mātugāmena saddhiṁ raho paṭicchanne āsane nisajjaṁ kappeyya, pācittiyaṁ.

\pdfbookmark[3]{Pācittiya 45}{pac45}
\subsubsection*{\hyperref[exp45]{Pācittiya 45: Rahonisajjasikkhāpadaṁ}}
\label{pac45}

Yo pana bhikkhu mātugāmena saddhiṁ eko ekāya raho nisajjaṁ kappeyya, pācittiyaṁ.

\pdfbookmark[3]{Pācittiya 46}{pac46}
\subsubsection*{\hyperref[exp46]{Pācittiya 46: Cārittasikkhāpadaṁ}}
\label{pac46}

Yo pana bhikkhu nimantito sabhatto samāno santaṁ bhikkhuṁ anāpucchā pure-bhattaṁ vā pacchā-bhattaṁ vā kulesu cārittaṁ āpajjeyya aññatra samayā, pācittiyaṁ. Tatth'āyaṁ samayo: cīvara-dāna-samayo, cīvara-kāra-samayo. Ayaṁ tattha samayo.

\pdfbookmark[3]{Pācittiya 47}{pac47}
\subsubsection*{\hyperref[exp47]{Pācittiya 47: Mahānāmasikkhāpadaṁ}}
\label{pac47}

Agilānena bhikkhunā cātu-māsa-paccaya-pavāraṇā sāditabbā, aññatra puna-pavāraṇāya, aññatra nicca-pavāraṇāya. Tato ce uttariṁ sādiyeyya, pācittiyaṁ.

\pdfbookmark[3]{Pācittiya 48}{pac48}
\subsubsection*{\hyperref[exp48]{Pācittiya 48: Uyyuttasenāsikkhāpadaṁ}}
\label{pac48}

Yo pana bhikkhu uyyuttaṁ senaṁ dassanāya gaccheyya, aññatra tathā-rūpa-paccayā, pācittiyaṁ.

\pdfbookmark[3]{Pācittiya 49}{pac49}
\subsubsection*{\hyperref[exp49]{Pācittiya 49: Senāvāsasikkhāpadaṁ}}
\label{pac49}

Siyā ca tassa bhikkhuno kocid'eva paccayo senaṁ gamanāya, dviratta-tirattaṁ tena bhikkhunā senāya vasitabbaṁ. Tato ce uttariṁ vaseyya, pācittiyaṁ.

\pdfbookmark[3]{Pācittiya 50}{pac50}
\subsubsection*{\hyperref[exp50]{Pācittiya 50: Uyyodhikasikkhāpadaṁ}}
\label{pac50}

Dviratta-tirattañ-ce bhikkhu senāya vasamāno, uyyodhikaṁ vā bal'aggaṁ vā senā-byūhaṁ vā anīka-dassanaṁ vā gaccheyya, pācittiyaṁ.

\begin{center}
  Acelaka-vaggo pañcamo
\end{center}

\subsection{Surāpānavaggo}
\vspace{0.2cm}

\pdfbookmark[3]{Pācittiya 51}{pac51}
\subsubsection*{\hyperref[exp51]{Pācittiya 51: Surāpānasikkhāpadaṁ}}
\label{pac51}

Surā-meraya-pāne pācittiyaṁ.

\pdfbookmark[3]{Pācittiya 52}{pac52}
\subsubsection*{\hyperref[exp52]{Pācittiya 52: Aṅgulipatodakasikkhāpadaṁ}}
\label{pac52}

Aṅguli-patodake pācittiyaṁ.

\pdfbookmark[3]{Pācittiya 53}{pac53}
\subsubsection*{\hyperref[exp53]{Pācittiya 53: Hassadhammasikkhāpadaṁ}}
\label{pac53}

Udake hassa-dhamme pācittiyaṁ.

\pdfbookmark[3]{Pācittiya 54}{pac54}
\subsubsection*{\hyperref[exp54]{Pācittiya 54: Anādariyasikkhāpadaṁ}}
\label{pac54}

Anādariye pācittiyaṁ.

\pdfbookmark[3]{Pācittiya 55}{pac55}
\subsubsection*{\hyperref[exp55]{Pācittiya 55: Bhiṁsāpanasikkhāpadaṁ}}
\label{pac55}

Yo pana bhikkhu bhikkhuṁ bhiṁsāpeyya, pācittiyaṁ.

\pdfbookmark[3]{Pācittiya 56}{pac56}
\subsubsection*{\hyperref[exp56]{Pācittiya 56: Jotikasikkhāpadaṁ}}
\label{pac56}

Yo pana bhikkhu agilāno visīvan'āpekkho, jotiṁ samādaheyya vā samādahāpeyya vā, aññatra tathā-rūpa-paccayā, pācittiyaṁ.

\pdfbookmark[3]{Pācittiya 57}{pac57}
\subsubsection*{\hyperref[exp57]{Pācittiya 57: Nahānasikkhāpadaṁ}}
\label{pac57}

Yo pana bhikkhu oren'aḍḍha-māsaṁ nhāyeyya, aññatra samayā, pācittiyaṁ. Tatth'āyaṁ samayo: “Diyaḍḍho māso seso gimhānan” ti, vassānassa paṭhamo māso, icc'ete aḍḍhateyya-māsā; uṇha-samayo, pariḷāha-samayo, gilāna-samayo, kamma-samayo, addhāna-gamana-samayo, vāta-vuṭṭhi-samayo. Ayaṁ tattha samayo.

\pdfbookmark[3]{Pācittiya 58}{pac58}
\subsubsection*{\hyperref[exp58]{Pācittiya 58: Dubbaṇṇakaraṇasikkhāpadaṁ}}
\label{pac58}

Navam-pana bhikkhunā cīvara-lābhena tiṇṇaṁ dubbaṇṇa-karaṇānaṁ aññataraṁ dubbaṇṇa-karaṇaṁ ādātabbaṁ, nīlaṁ vā kaddamaṁ vā kāḷa-sāmaṁ vā. Anādā ce bhikkhu tiṇṇaṁ dubbaṇṇa-karaṇānaṁ aññataraṁ dubbaṇṇa-karaṇaṁ navaṁ cīvaraṁ paribhuñjeyya, pācittiyaṁ.

\pdfbookmark[3]{Pācittiya 59}{pac59}
\subsubsection*{\hyperref[exp59]{Pācittiya 59: Vikappanasikkhāpadaṁ}}
\label{pac59}

Yo pana bhikkhu bhikkhussa vā bhikkhuniyā vā sikkhamānāya vā sāmaṇerassa vā sāmaṇeriyā vā sāmaṁ cīvaraṁ vikappetvā apaccuddhārakaṁ paribhuñjeyya, pācittiyaṁ.

\pdfbookmark[3]{Pācittiya 60}{pac60}
\subsubsection*{\hyperref[exp60]{Pācittiya 60: Apanidhānasikkhāpadaṁ}}
\label{pac60}
Yo pana bhikkhu bhikkhussa pattaṁ vā cīvaraṁ vā nisīdanaṁ vā sūci-gharaṁ vā kāya-bandhanaṁ vā apanidheyya vā apanidhāpeyya vā, antamaso hass'āpekkho-pi, pācittiyaṁ.

\begin{center}
  Surā-pāna-vaggo chaṭṭho
\end{center}

\subsection{Sappāṇavaggo}
\vspace{0.2cm}

\pdfbookmark[3]{Pācittiya 61}{pac61}
\subsubsection*{\hyperref[exp61]{Pācittiya 61: Sañciccasikkhāpadaṁ}}
\label{pac61}

Yo pana bhikkhu sañcicca pāṇaṁ jīvitā voropeyya, pācittiyaṁ.

\pdfbookmark[3]{Pācittiya 62}{pac62}
\subsubsection*{\hyperref[exp62]{Pācittiya 62: Sappāṇakasikkhāpadaṁ}}
\label{pac62}

Yo pana bhikkhu jānaṁ sappāṇakaṁ udakaṁ paribhuñjeyya, pācittiyaṁ.

\pdfbookmark[3]{Pācittiya 63}{pac63}
\subsubsection*{\hyperref[exp63]{Pācittiya 63: Ukkoṭanasikkhāpadaṁ}}
\label{pac63}

Yo pana bhikkhu jānaṁ yathādhammaṁ nīhatādhikaraṇaṁ punakammāya ukkoṭeyya, pācittiyaṁ.

\pdfbookmark[3]{Pācittiya 64}{pac64}
\subsubsection*{\hyperref[exp64]{Pācittiya 64: Duṭṭhullasikkhāpadaṁ}}
\label{pac64}

Yo pana bhikkhu bhikkhussa jānaṁ duṭṭhullaṁ āpattiṁ paṭicchādeyya, pācittiyaṁ.

\pdfbookmark[3]{Pācittiya 65}{pac65}
\subsubsection*{\hyperref[exp65]{Pācittiya 65: Ūnavīsativassasikkhāpadaṁ}}
\label{pac65}

Yo pana bhikkhu jānaṁ ūna-vīsati-vassaṁ puggalaṁ upasampādeyya, so ca puggalo anupasampanno, te ca bhikkhū gārayhā. Idaṁ tasmiṁ pācittiyaṁ.

\pdfbookmark[3]{Pācittiya 66}{pac66}
\subsubsection*{\hyperref[exp66]{Pācittiya 66: Theyyasatthasikkhāpadaṁ}}
\label{pac66}

Yo pana bhikkhu jānaṁ theyya-satthena saddhiṁ saṁvidhāya ek'addhāna-maggaṁ paṭipajjeyya, antamaso gām'antaram-pi, pācittiyaṁ.

\pdfbookmark[3]{Pācittiya 67}{pac67}
\subsubsection*{\hyperref[exp67]{Pācittiya 67: Saṁvidhānasikkhāpadaṁ}}
\label{pac67}

Yo pana bhikkhu mātugāmena saddhiṁ saṁvidhāya ek'addhāna-maggaṁ paṭipajjeyya, antamaso gām'antaram-pi, pācittiyaṁ.

\pdfbookmark[3]{Pācittiya 68}{pac68}
\subsubsection*{\hyperref[exp68]{Pācittiya 68: Ariṭṭhasikkhāpadaṁ}}
\label{pac68}

Yo pana bhikkhu evaṁ vadeyya, “Tath'āhaṁ bhagavatā dhammaṁ desitaṁ ājānāmi, yathā ye'me antarāyikā dhammā vuttā bhagavatā, te paṭisevato n'ālaṁ antarāyāyā” ti. So bhikkhu bhikkhūhi evam'assa vacanīyo, “Mā āyasmā evaṁ avaca. Mā bhagavantaṁ abbhācikkhi. Na hi sādhu bhagavato abbhakkhānaṁ. Na hi bhagavā evaṁ vadeyya. Aneka-pariyāyena āvuso antarāyikā dhammā vuttā bhagavatā, alañ-ca pana te paṭisevato antarāyāyā” ti. Evañ-ca so bhikkhu bhikkhūhi vuccamāno tath'eva paggaṇheyya, so bhikkhu bhikkhūhi yāva-tatiyaṁ samanubhāsitabbo tassa paṭinissaggāya. Yāva-tatiyañ-ce samanubhāsiyamāno taṁ paṭinissajjeyya, icc'etaṁ kusalaṁ. No ce paṭinissajjeyya, pācittiyaṁ.

\pdfbookmark[3]{Pācittiya 69}{pac69}
\subsubsection*{\hyperref[exp69]{Pācittiya 69: Ukkhittasambhogasikkhāpadaṁ}}
\label{pac69}

Yo pana bhikkhu jānaṁ tathā-vādinā bhikkhunā akaṭ'ānudhammena taṁ diṭṭhiṁ appaṭinissaṭṭhena, saddhiṁ sambhuñjeyya vā saṁvaseyya vā saha vā seyyaṁ kappeyya, pācittiyaṁ.

\pdfbookmark[3]{Pācittiya 70}{pac70}
\subsubsection*{\hyperref[exp70]{Pācittiya 70: Kaṇṭakasikkhāpadaṁ}}
\label{pac70}

Samaṇuddeso'pi ce evaṁ vadeyya, “Tath'āhaṁ bhagavatā dhammaṁ desitaṁ ājānāmi, yathā ye'me antarāyikā dhammā vuttā bhagavatā, te paṭisevato n'ālaṁ antarāyāyā” ti. So samaṇuddeso bhikkhūhi evam'assa vacanīyo, “Mā āvuso samaṇuddesa evaṁ avaca. Mā bhagavantaṁ abbhācikkhi. Na hi sādhu bhagavato abbhakkhānaṁ. Na hi bhagavā evaṁ vadeyya. Aneka-pariyāyena āvuso samaṇuddesa antarāyikā dhammā vuttā bhagavatā, alañ-ca pana te paṭisevato antarāyāyā” ti. Evañ-ca so samaṇuddeso bhikkhūhi vuccamāno tath'eva paggaṇheyya, so samaṇuddeso bhikkhūhi evam'assa vacanīyo, “Ajjatagge te āvuso samaṇuddesa na c'eva so bhagavā satthā apadisitabbo, yam'pi c'aññe samaṇuddesā labhanti bhikkhūhi saddhiṁ dviratta-tirattaṁ saha-seyyaṁ, sā'pi te n'atthi. Cara'pi re vinassā” ti. Yo pana bhikkhu jānaṁ tathā-nāsitaṁ samaṇuddesaṁ upalāpeyya vā upaṭṭhāpeyya vā sambhuñjeyya vā saha vā seyyaṁ kappeyya, pācittiyaṁ.

\begin{center}
  Sappāṇa-vaggo sattamo
\end{center}

\subsection{Sahadhammikavaggo}
\vspace{0.2cm}

\pdfbookmark[3]{Pācittiya 71}{pac71}
\subsubsection*{\hyperref[exp71]{Pācittiya 71: Sahadhammikasikkhāpadaṁ}}
\label{pac71}

Yo pana bhikkhu bhikkhūhi saha-dhammikaṁ vuccamāno evaṁ vadeyya, “Na tāv'āhaṁ āvuso etasmiṁ sikkhāpade sikkhissāmi, yāva n'aññaṁ bhikkhuṁ byattaṁ vinaya-dharaṁ paripucchāmī” ti, pācittiyaṁ. Sikkhamānena bhikkhave bhikkhunā aññātabbaṁ paripucchitabbaṁ paripañhitabbaṁ. Ayaṁ tattha sāmīci.

\pdfbookmark[3]{Pācittiya 72}{pac72}
\subsubsection*{\hyperref[exp72]{Pācittiya 72: Vilekhanasikkhāpadaṁ}}
\label{pac72}

Yo pana bhikkhu pāṭimokkhe uddissamāne evaṁ vadeyya, “Kim-pan'imehi khudd'ānukhuddakehi sikkhāpadehi uddiṭṭhehi, yāvad'eva kukkuccāya vihesāya vilekhāya saṁvattantī” ti. Sikkhāpada-vivaṇṇanake, pācittiyaṁ.

\pdfbookmark[3]{Pācittiya 73}{pac73}
\subsubsection*{\hyperref[exp73]{Pācittiya 73: Mohanasikkhāpadaṁ}}
\label{pac73}

Yo pana bhikkhu anvaḍḍha-māsaṁ pāṭimokkhe uddissamāne evaṁ vadeyya, “Idān'eva kho ahaṁ ājānāmi, ‘Ayam'pi kira dhammo sutt'āgato sutta-pariyāpanno anvaḍḍha-māsaṁ uddesaṁ āgacchatī'” ti. Tañ-ce bhikkhuṁ aññe bhikkhū jāneyyuṁ, “Nisinna-pubbaṁ iminā bhikkhunā dvittikkhattuṁ pāṭimokkhe uddissamāne, ko pana vādo bhiyyo” ti, na ca tassa bhikkhuno aññāṇakena mutti atthi. Yañ-ca tattha āpattiṁ āpanno, tañ-ca yathā-dhammo kāretabbo, uttariñ-c'assa moho āropetabbo, “Tassa te āvuso alābhā, tassa te dulladdhaṁ, yaṁ tvaṁ pāṭimokkhe uddissamāne na sādhukaṁ aṭṭhikatvā manasikarosī” ti. Idaṁ tasmiṁ mohanake, pācittiyaṁ.

\pdfbookmark[3]{Pācittiya 74}{pac74}
\subsubsection*{\hyperref[exp74]{Pācittiya 74: Pahārasikkhāpadaṁ}}
\label{pac74}

Yo pana bhikkhu bhikkhussa kupito anattamano pahāraṁ dadeyya, pācittiyaṁ.

\pdfbookmark[3]{Pācittiya 75}{pac75}
\subsubsection*{\hyperref[exp75]{Pācittiya 75: Talasattikasikkhāpadaṁ}}
\label{pac75}

Yo pana bhikkhu bhikkhussa kupito anattamano talasattikaṁ uggireyya, pācittiyaṁ.

\pdfbookmark[3]{Pācittiya 76}{pac76}
\subsubsection*{\hyperref[exp76]{Pācittiya 76: Amūlakasikkhāpadaṁ}}
\label{pac76}

Yo pana bhikkhu bhikkhuṁ amūlakena saṅghādisesena anuddhaṁseyya, pācittiyaṁ.

\pdfbookmark[3]{Pācittiya 77}{pac77}
\subsubsection*{\hyperref[exp77]{Pācittiya 77: Sañciccasikkhāpadaṁ}}
\label{pac77}

Yo pana bhikkhu bhikkhussa sañcicca kukkuccaṁ upadaheyya, “Iti'ssa muhuttam'pi aphāsu bhavissatī” ti. Etad'eva paccayaṁ karitvā anaññaṁ, pācittiyaṁ.

\pdfbookmark[3]{Pācittiya 78}{pac78}
\subsubsection*{\hyperref[exp78]{Pācittiya 78: Upassutisikkhāpadaṁ}}
\label{pac78}

Yo pana bhikkhu bhikkhūnaṁ bhaṇḍanajātānaṁ kalahajātānaṁ vivād'āpannānaṁ upassutiṁ tiṭṭheyya, “Yaṁ ime bhaṇissanti taṁ sossāmī” ti. Etad'eva paccayaṁ karitvā anaññaṁ, pācittiyaṁ.

\pdfbookmark[3]{Pācittiya 79}{pac79}
\subsubsection*{\hyperref[exp79]{Pācittiya 79: Kammappaṭibāhanasikkhāpadaṁ}}
\label{pac79}

Yo pana bhikkhu dhammikānaṁ kammānaṁ chandaṁ datvā, pacchā khiyyana-dhammaṁ āpajjeyya, pācittiyaṁ.

\pdfbookmark[3]{Pācittiya 80}{pac80}
\subsubsection*{\hyperref[exp80]{Pācittiya 80: Chandaṁ-adatvā-gamanasikkhāpadaṁ}}
\label{pac80}

Yo pana bhikkhu saṅghe vinicchaya-kathāya vattamānāya, chandaṁ adatvā uṭṭhāy‘āsanā pakkameyya, pācittiyaṁ.

\pdfbookmark[3]{Pācittiya 81}{pac81}
\subsubsection*{\hyperref[exp81]{Pācittiya 81: Dubbalasikkhāpadaṁ}}
\label{pac81}

Yo pana bhikkhu samaggena saṅghena cīvaraṁ datvā, pacchā khiyyana-dhammaṁ āpajjeyya, “Yathā-santhutaṁ bhikkhū saṅghikaṁ lābhaṁ pariṇāmentī” ti, pācittiyaṁ.

\pdfbookmark[3]{Pācittiya 82}{pac82}
\subsubsection*{\hyperref[exp82]{Pācittiya 82: Pariṇāmanasikkhāpadaṁ}}
\label{pac82}

Yo pana bhikkhu jānaṁ saṅghikaṁ lābhaṁ pariṇataṁ puggalassa pariṇāmeyya, pācittiyaṁ.

\begin{center}
  Sahadhammika-vaggo aṭṭhamo
\end{center}

\subsection{Rājavaggo}
\vspace{0.2cm}

\pdfbookmark[3]{Pācittiya 83}{pac83}
\subsubsection*{\hyperref[exp83]{Pācittiya 83: Antepurasikkhāpadaṁ}}
\label{pac83}

Yo pana bhikkhu rañño khattiyassa muddh'ābhisittassa anikkhanta-rājake aniggata-ratanake pubbe appaṭisaṁvidito indakhīlaṁ atikkāmeyya, pācittiyaṁ.

\pdfbookmark[3]{Pācittiya 84}{pac84}
\subsubsection*{\hyperref[exp84]{Pācittiya 84: Ratanasikkhāpadaṁ}}
\label{pac84}
Yo pana bhikkhu ratanaṁ vā ratana-sammataṁ vā aññatra ajjhārāmā vā ajjhāvasathā vā uggaṇheyya vā uggaṇhāpeyya vā, pācittiyaṁ. Ratanaṁ vā pana bhikkhunā ratana-sammataṁ vā, ajjhārāme vā ajjhāvasathe vā uggahetvā vā uggaṇhāpetvā vā nikkhipitabbaṁ, “Yassa bhavissati so harissatī” ti. Ayaṁ tattha sāmīci.

\pdfbookmark[3]{Pācittiya 85}{pac85}
\subsubsection*{\hyperref[exp85]{Pācittiya 85: Vikālagāmappavesanasikkhāpadaṁ}}
\label{pac85}

Yo pana bhikkhu santaṁ bhikkhuṁ anāpucchā vikāle gāmaṁ paviseyya, aññatra tathā-rūpā accāyikā karaṇīyā, pācittiyaṁ.

\pdfbookmark[3]{Pācittiya 86}{pac86}
\subsubsection*{\hyperref[exp86]{Pācittiya 86: Sūcigharasikkhāpadaṁ}}
\label{pac86}

Yo pana bhikkhu aṭṭhi-mayaṁ vā danta-mayaṁ vā visāṇa-mayaṁ vā sūci-gharaṁ kārāpeyya, bhedanakaṁ pācittiyaṁ.

\pdfbookmark[3]{Pācittiya 87}{pac87}
\subsubsection*{\hyperref[exp87]{Pācittiya 87: Mañcapīṭhasikkhāpadaṁ}}
\label{pac87}

Navam-pana bhikkhunā mañcaṁ vā pīṭhaṁ vā kārayamānena, aṭṭh'aṅgula-pādakaṁ kāretabbaṁ sugat'aṅgulena, aññatra heṭṭhimāya aṭaniyā. Taṁ atikkāmayato, chedanakaṁ pācittiyaṁ.

\pdfbookmark[3]{Pācittiya 88}{pac88}
\subsubsection*{\hyperref[exp88]{Pācittiya 88: Tūlonaddhasikkhāpadaṁ}}
\label{pac88}

Yo pana bhikkhu mañcaṁ vā pīṭhaṁ vā tūlonaddhaṁ kārāpeyya, uddālanakaṁ pācittiyaṁ.

\pdfbookmark[3]{Pācittiya 89}{pac89}
\subsubsection*{\hyperref[exp89]{Pācittiya 89: Nisīdanasikkhāpadaṁ}}
\label{pac89}

Nisīdanam-pana bhikkhunā kārayamānena pamāṇikaṁ kāretabbaṁ. Tatr'idaṁ pamāṇaṁ: dīghaso dve vidatthiyo sugata-vidatthiyā, tiriyaṁ diyaḍḍhaṁ, dasā vidatthi. Taṁ atikkāmayato, chedanakaṁ pācittiyaṁ.

\pdfbookmark[3]{Pācittiya 90}{pac90}
\subsubsection*{\hyperref[exp90]{Pācittiya 90: Kaṇḍuppaṭicchādisikkhāpadaṁ}}
\label{pac90}

Kaṇḍu-paṭicchādiṁ pana bhikkhunā kārayamānena pamāṇikā kāretabbā. Tatr'idaṁ pamāṇaṁ: dīghaso catasso vidatthiyo sugata-vidatthiyā, tiriyaṁ dve vidatthiyo. Taṁ atikkāmayato, chedanakaṁ pācittiyaṁ.

\pdfbookmark[3]{Pācittiya 91}{pac91}
\subsubsection*{\hyperref[exp91]{Pācittiya 91: Vassikasāṭikasikkhāpadaṁ}}
\label{pac91}

Vassika-sāṭikaṁ pana bhikkhunā kārayamānena pamāṇikā kāretabbā. Tatr'idaṁ pamāṇaṁ: dīghaso cha vidatthiyo sugata-vidatthiyā tiriyaṁ aḍḍha-teyyā. Taṁ atikkāmayato, chedanakaṁ pācittiyaṁ.

\pdfbookmark[3]{Pācittiya 92}{pac92}
\subsubsection*{\hyperref[exp92]{Pācittiya 92: Nandasikkhāpadaṁ}}
\label{pac92}

Yo pana bhikkhu sugata-cīvarappamāṇaṁ cīvaraṁ kārāpeyya atirekaṁ vā, chedanakaṁ pācittiyaṁ. Tatr'idaṁ sugatassa sugata-cīvarappamāṇaṁ: dīghaso nava vidatthiyo sugata-vidatthiyā, tiriyaṁ cha vidatthiyo. Idaṁ sugatassa sugata-cīvarappamāṇaṁ.

\begin{center}
  Ratana-vaggo navamo
\end{center}

\medskip

\begin{center}
Uddiṭṭhā kho āyasmanto dve-navuti pācittiyā dhammā.

\smallskip

Tatth'āyasmante pucchāmi: Kacci'ttha parisuddhā?\\
Dutiyam'pi pucchāmi: Kacci'ttha parisuddhā?\\
Tatiyam'pi pucchāmi: Kacci'ttha parisuddhā?

\smallskip

Parisuddh'etth'āyasmanto, tasmā tuṇhī, evam'etaṁ dhārayāmi.
\end{center}

\begin{outro}
  Pācittiyā niṭṭhitā
\end{outro}

\clearpage
