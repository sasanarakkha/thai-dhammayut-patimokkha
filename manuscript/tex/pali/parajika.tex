\section{Pārājik'uddeso}
\label{par}

\begin{intro}
  Tatr'ime cattāro pārājikā dhammā uddesaṁ āgacchanti.
\end{intro}

\setsubsecheadstyle{\subsubsectionFmt}
\pdfbookmark[2]{Pārājika 1}{par1}
\subsection*{\hyperref[disq1]{Pārājika 1: Methunadhammasikkhāpadaṁ}}
\label{par1}

Yo pana bhikkhu bhikkhūnaṁ sikkhā-sājīva-samāpanno, sikkhaṁ appaccakkhāya dubbalyaṁ anāvikatvā, methunaṁ dhammaṁ paṭiseveyya antamaso tiracchāna-gatāya'pi: pārājiko hoti asaṁvāso.

\pdfbookmark[2]{Pārājika 2}{par2}
\subsection*{\hyperref[disq2]{Pārājika 2: Adinn'ādānasikkhāpadaṁ}}
\label{par2}

Yo pana bhikkhu gāmā vā araññā vā adinnaṁ theyyasaṅkhātaṁ ādiyeyya, yathārūpe adinn'ādāne rājāno coraṁ gahetvā haneyyuṁ vā bandheyyuṁ vā pabbājeyyuṁ vā: ``Coro'si, bālo'si, mūḷho'si, theno'sī'ti,'' tathārūpaṁ bhikkhu adinnaṁ ādiyamāno; ayam'pi pārājiko hoti, asaṁvāso.

\pdfbookmark[2]{Pārājika 3}{par3}
\subsection*{\hyperref[disq3]{Pārājika 3: Manussaviggahasikkhāpadaṁ}}
\label{par3}

Yo pana bhikkhu sañcicca manussa-viggahaṁ jīvitā voropeyya, satthahārakaṁ vā'ssa pariyeseyya, maraṇa-vaṇṇaṁ vā saṁvaṇṇeyya, maraṇāya vā samādapeyya, “Ambho purisa kiṁ tuyh'iminā pāpakena dujjīvitena? Matan-te jīvitā seyyo”ti. Iti cittamano citta-saṅkappo anekapariyāyena maraṇa-vaṇṇaṁ vā saṁvaṇṇeyya, maraṇāya vā samādapeyya: ayam'pi pārājiko hoti asaṁvāso.

\pdfbookmark[2]{Pārājika 4}{par4}
\subsection*{\hyperref[disq4]{Pārājika 4: Uttarimanussadhammasikkhāpadaṁ}}
\label{par4}

Yo pana bhikkhu anabhijānaṁ uttari-manussa-dhammaṁ attūpanāyikaṁ alam-ariya-ñāṇa-dassanaṁ samudācareyya: “Iti jānāmi, iti passāmī”ti. Tato aparena samayena samanuggāhiyamāno vā asamanuggāhiyamāno vā āpanno visuddh'āpekkho evaṁ vadeyya, “Ajānam-evaṁ āvuso avacaṁ, ‘jānāmi,' apassaṁ, ‘passāmi.' Tucchaṁ musā vilapin”ti. Aññatra adhimānā: ayam'pi pārājiko hoti asaṁvāso.

\medskip

\begin{center}
Uddiṭṭhā kho āyasmanto cattāro pārājikā dhammā, yesaṁ bhikkhu aññataraṁ vā aññataraṁ vā āpajjitvā na labhati bhikkhūhi saddhiṁ saṁvāsaṁ. Yathā pure, tathā pacchā: pārājiko hoti asaṁvāso.

\smallskip

Tatth'āyasmante pucchāmi: Kacci'ttha parisuddhā?\\
Dutiyam'pi pucchāmi: Kacci'ttha parisuddhā?\\
Tatiyam'pi pucchāmi: Kacci'ttha parisuddhā?

\smallskip

Parisuddh'etth'āyasmanto, tasmā tuṇhī, evam'etaṁ dhārayāmi.
\end{center}

\begin{outro}
  Pārājik'uddeso niṭṭhito
\end{outro}

\clearpage
